\documentclass{article}
\usepackage[utf8]{inputenc}
\usepackage[margin=1in]{geometry}
\usepackage{enumerate}
\usepackage{amsmath}
\usepackage{amstext}
\usepackage{array}
\newcommand{\thf}{\rule{\textwidth}{.4pt}}

\title{Submission 2.2}
\author{Jonas Wechsler}
\date{October 2014}

\begin{document}
	\maketitle
	\section{Problems 2.8.3}
	\begin{enumerate}
		\item
			Since $f(x) = ax^2 + bx + c$ is a quadratic function, then $\frac{-b \pm \sqrt{b^2-4ac}}{2a} = x_0$ where $f(x_0) = 0$. $\frac{-b \pm \sqrt{b^2-4ac}}{2a} = x_0 = \pm 1$, so $f(\pm 1) = 0$.
		\item
			The conclusion is stated  
		\item
			$S_{xy}$: x is continuous at y\\
			$f:$ our fcn\\
			$c:$ the point in question (in this case, 0)\\
			$C_{xy}:$ x converges to y\\
			\\
			\begin{tabular}{l l l}
				$S_{fc}$ & $\iff \forall _x (C_{xc} \implies C_{f(x)f(c)})$ & Our definition of continuous.\\
				$\lnot S_{fc}$ & $\iff \lnot \forall _x (C_{xc} \implies C_{f(x)f(c)})$ & Our definition of not continuous.\\
				& $\iff \exists _x \lnot(\lnot C_{xc} \lor C_{f(x)f(c)})$\\
				& $\iff \exists _x (C_{xc} \land \lnot C_{f(x)f(c)})$\\
			\end{tabular}\\
			\\
			Therefore, we need to show that there exists a function (call it $g(x)$) that does not converge at 0 and that our function (call it $f(x)$) does not converge at f(g(
		\item
	\end{enumerate}
	\section{Arguments in English 2.8.4}
	\begin{enumerate}
		\item
			\begin{enumerate}
				\item[b]
					Invalid\\
					Unsound	
				\item[d]
					$C_{xy}$: x is caused by y\\ 
					\begin{tabular}{l l}
						1. $\forall _y \exists _x (C_{xy})$\\
						2. $\exists _y \forall _x (C_{xy})$\\
					\end{tabular}
			\end{enumerate}
		\item
			\begin{enumerate}
				\item[b]
					Valid\\
					Invalid
				\item[d]
					$H_{xy}$: x hates y\\ 
					$f$: Fred\\
					$a$: Al\\
					\begin{tabular}{l l}
						1. $\forall _x (H_{xa} \implies H_{fx})$ & $P_1$\\
						2. $\forall _x H_{ax}$ & $P_1$\\ 
						3. $H_{aa}$ & UI(2)\\
						4. $H_{aa} \implies H_{fa}$ & UI(1)\\
						6. $H_{fa}$ & MP(3,4) \\
						7. $H_{af}$ & UI(2) \\
						8. $H_{af} \land H_{fa}$ & Conj(6,7)\\
					\end{tabular}

			\end{enumerate}
		\item
			\begin{enumerate}
				\item[b]
					Valid\\
					Soundness is difficult to determine
				\item[d]
					$L_x$: x is large and hostile\\
					$I_x$: x is impervious to pesticides\\
					$i$: An insect in this house\\
					\begin{tabular}{l l}
						1. $\forall _i (L_i)$ & $P_1$\\
						2. $\exists _i (I_i)$ & $P_2$\\
						3. $L_x$ & $UI(1)$\\
						4. $I_x$ & $EI(2)$ \\
						5. $L_x \land I_x$ & $Conj(3,4)$\\
						6. $\exists _i (L_i \land I_i)$ & EG(5)
					\end{tabular}
			\end{enumerate}
		\item
			\begin{enumerate}
				\item[b]
					Valid\\
					Sound
				\item[d]
					$S_x$: x can succeed at the university\\
					$B_x$: x is bright\\
					$M_x$: x is mature\\
					$s$: student\\
					\begin{tabular}{l l}
						1. $\exists _s \lnot S_s$ & $P_1$\\
						2. $(B_s \land M_x) \implies (S_s)$ & $P_2$\\
						3. $\lnot (S_s) \implies \lnot(B_s \land M_x)$ & CP(2)\\
						4. $\lnot (S_s) \implies \lnot B_s \lor \lnot M_x$ & DM(3)\\
						5. $\lnot S_x$ & EI(1)\\
						6. $\lnot B_x \lor \lnot M_x$ & MP(4,5)\\
						7. $\exists _x (\lnot B_x \lor \lnot M_x)$ & EG(6)\\
					\end{tabular}
			\end{enumerate}
		\item
			\begin{enumerate}
				\item[b]
					Valid\\
					Sound
				\item[d]
					$P_x$: x is a pig\\
					\begin{tabular}{l l}
						1. $\exists _x \exists _y \exists _z ((x \neq y)\land(x \neq z)\land(z \neq y)\land P_x \land P_y \land P_z)$ & $P_1$\\
						2. $\exists _x \exists _y ((x \neq y) \land (x \neq a) \land (a \neq y) \land P_x \land P_y \land P_a)$ & $EI(1)$\\
						3. $\exists _x \exists _y ((x \neq y)\land P_x \land P_y)$ & Simp(2)\\
					\end{tabular}
			\end{enumerate}
		\item
			\begin{enumerate}
				\item[b]
					Valid
					Sound
				\item[d]
					$L_{xy}$: x likes y\\
					$p$: Popeye\\
					$o$: Olive Oyl\\
					\begin{tabular}{l l}
						1. $\forall x (L_{xo} \implies L_{px})$ & $P_1$\\
						2. $\forall x (L_{ox})$ & $P_2$\\
						3. $L_{oo}$ & UI(2)\\
						4. $L_{oo} \implies L_{po}$ & UI(1)\\
						5. $L_{po}$ & MP(3,4)\\
						6. $L_{op}$ & UI(2)\\
						6. $L_{po} \land L_{op}$ & Conj(5,6)\\
					\end{tabular}
			\end{enumerate}
		\item
			\begin{enumerate}
				\item[a]
					$S_x$: x is sound\\
					$C_x$: x has a true conclusion\\
					$a$: this argument\\
					\begin{tabular}{l}
						$\lnot C_a$\\
						$\lnot \exists _x (S_x \land \lnot C_x)$\\ \hline
						$\lnot S_a$\\
					\end{tabular}
				\item[b]
					Valid\\
					Sound
				\item[c]
					Assume there is an interpretation in which $\lnot C_x$ and $\lnot \exists _x (S_x \land \lnot C_x)$ are both true, but $\lnot S_x$ is false. So there must be some element of UD, call it 1, such that $(1 \ni S)$ and $\lnot (1 \ni C)$. However, the second premise would be false in this interpretation. $\rightarrow \leftarrow$ No such interpretation.i
				\item[d]
					$S_x$: x is sound\\
					$C_x$: x has a true conclusion\\
					$a$: this argument\\
					\begin{tabular}{l l}
						1. $\lnot C_a$ & $P_1$\\
						2. $\lnot \exists _x (S_x \land \lnot C_x)$ & $P_2$\\
						3. $\forall _x \lnot (S_x \land \lnot C_x)$ & QEx(2)\\
						4. $\forall _x (\lnot S_x \lor \lnot \lnot C_x)$ & DM(3)\\
						5. $\forall _x (\lnot S_x \lor C_x)$ & DN(4)\\
						6. $\lnot S_a \lor C_a$ & UI(5)\\
						7. $S_a \implies C_a$ & CSis(6)\\
						8. $\lnot C_a \implies \lnot S_a$ & CP(7)\\
						4. $\lnot S_a$ & MP(1,8)\\
					\end{tabular}
			\end{enumerate}
		\item
			\begin{enumerate}
				\item[a]
					$O_x$: x weighs over 200 pounds\\
					$j$: Jones' killer\\
					$s$: Smith\\
					\begin{tabular}{l}
						$O_j$\\
						$\lnot O_s$\\ \hline
						$O_s \neq O_j$\\
					\end{tabular}
				\item[c]
					Assume there is an interpretation in which $O_x$ and $\lnot O_y$, but not $O_y \neq O_x$. So there must be two elements of UD, call them 1 and 2, such that $1 \ni O$ and $\lnot (2 \ni O)$, and $2 = 1$. 
				\item[d]
					$O_x$: x weighs over 200 pounds\\
					$j$: Jones' killer\\
					$s$: Smith\\
					\begin{tabular}{l l}
						$O_j$ & $P_1$\\
						$\lnot O_s$ & $P_2$\\
						$\lnot (O_s \iff O_j)$ & \\
						$\forall _x \forall _y(x=y \implies (P_x \iff P_y))$ & Leibniz\\
						$\forall _x \forall _y(\lnot (P_x \iff P_y) \implies x \neq y)$ & CP\\
						$O_s \neq O_j$ & MP\\
					\end{tabular}
			\end{enumerate}
		\item
			\begin{enumerate}
				\item[b]
					Valid\\
					Not sound	
				\item[d]
					$L_{xy}$: x likes y\\ 
					$m$: mandy\\
					$a$: andy\\
					\begin{tabular}{l l}
						1. $\forall _x(L_{xm})$ & $P_1$\\
						2. $\forall _x(L_{mx} \iff (m = a))$ & $P_2$\\
						3. $L_{mm}$ & UI(1)\\
						4. $L_{mm} \iff (m = a)$ & UI(2)\\
						5. $(L_{mm} \implies (m = a)) \land ((m = a) \implies L_{mm})$ & Equiv(4)\\
						6. $L_{mm} \implies (m = a)$ & Simp(5)\\
						7. $m = a$ & MP(3,6)
					\end{tabular}

			\end{enumerate}
		\item
			\begin{enumerate}
				\item[b]
					Valid\\
					Sound
				\item[d]
					$A_{xy}$: x if afraid of y\\ 
					$h$: Mr. Hyde\\
					$j$: Dr. Jekyll\\
					\begin{tabular}{l l}
						1. $\forall _x (A_{xh})$\\
						2. $\forall _x (A_{hx} \iff (x = j))$\\
						3. $A_{hh}$ & UI(1)\\
						4. $A_{hh} \iff (h = j)$ & UI(2)\\
						5. $(L_{hh} \implies (h = j)) \land ((h = j) \implies L_{hh})$ & Equiv(4)\\
						6. $L_{hh} \implies (h = j)$ & Simp(5)\\
						7. $j = h$
					\end{tabular}
			\end{enumerate}
	\end{enumerate}
	\section{Arguments in quantificational logic 2.8.5}
	\begin{enumerate}
		\item[2.]
			\begin{enumerate}
				\item[b]
					Assume the premise is true. In order for it to be true in all cases that an element of F must be within G, all elements of F must be contained within G. Therefore, if an element is not in G, it cannot be within F.
				\item[c]
					\begin{tabular}{l l}
				1. $\forall _x(F_x \implies G_x)$ & $P_1$\\
				2. $\forall _x(\lnot G_x \implies \lnot F_x) & MP(1)\\
			\end{tabular}
	\end{enumerate}
\end{enumerate}
\end{document}

