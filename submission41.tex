\documentclass{article}
\usepackage[utf8]{inputenc}
\usepackage[margin=1in]{geometry}
\usepackage{enumerate}
\usepackage{amsmath}
\usepackage{amssymb}
\usepackage{amstext}
\usepackage{array}
\newcommand{\thf}{\rule{\textwidth}{.4pt}}

\title{Submission 4.1}
\author{Jonas Wechsler}
\date{November 2014}

\begin{document}
	\maketitle
	\begin{enumerate}
		\item
			$P_0:$ If only one student has a black card, they know that they are the only one with a black card because they do not see anybody else with a black card.\\
			Assume that it takes exactly k repetitions for k students with black cards.\\
			Consider $k+1$ students with black cards, and an individual $A$ who has a black cards. After $k$ turns, all the other students with black cards say nothing, so $A$ can infer that he has a black card, and will say yes on the next turn.
		\item
			$P_0: 1^3 + 5*1 = 6, \frac{6}{6} = 1 \land 1 \epsilon \mathbb{N}$\\
			$P_k:$ Assume $k^3 + 5k$ is divisible by 6.\\
			\begin{tabular}{l l}
				$P_{k+1}:$ & $ (k+1)^3 + 5(k+1)$\\
				& $ k^3 + 3k^2 + 3k + 1 + 5k + 5$\\
				& $ k^3 + 3k^2 + 8k + 6$\\
				& $ (k^3 + 5k) + 3k(k + 1) + 6$\\
			\end{tabular}\\
			A sum of numbers that are multiples of 6 is also a multiple a multiple of 6. We have already assumed that $k^3 + 5k$ is a multiple of 6 and we have already shown that 6 is a multiple of 6. If k is even, then $3k$ is a multiple of 6, so $3k(k + 1)$ is also a multiple of 6. If k is odd, then $k+1$ is even and $3(k+1)$ is a multiple of 6. So, $3k(k+1)$ is a multiple of 6. Because the expression $(k^3 + 5k) + 3k(k+1) + 6$ is a multiple of 6, it is also divisible by 6.
		\item
			$P_0:$ 6 can be represented with 3 2 cent coins.\\
			Assume n cents can be represented with 2 and 7 cent coins.\\
			There are 2 cases for n+1. If n+1 is even, it can be represented with 2 cent coins. If n+1 is odd, then n+1-7 is even, and can be represented with 2 cent coins and a 7 cent coin.\\
		\item
			$P_0: |1| \leq |1|$\\
			$P_k:$ Assume $|x_1 + x_2 + ... + x_n| \leq |x_1| + |x_2| + ... + |x_n|$\\
			$P_{k+1}: |x_1 + x_2 + ... + x_n + x_{n+1}| \leq |x_1| + |x_2| + ... + |x_n| + |x_{n+1}|$\\
			In the case that $x_{n+1}$ is negative, $|x_1 + ... + x_n| \geq |x_1 + ... + x_{n+1}|$, and $|x_1| + ... + |x_n| \leq |x_1| + ... + |x_{n+1}|$, meaning that $|x_1 + ... + x_{n+1}| \leq |x+1| + ... + |x_{n+1}|$. In the case that $x_{n+1}$ is zero, $P_{k+1} = P_{k}$. In the case that $x_{n+1}$ is positive, $|x_1 + ... + x_n| + |x_{n+1}| = |x_1 + ... + x_{n+1}|$, so $|x_1 + ... + x_{n+1}| \leq |x_1| + ... + |x_{n+1}|$.
		\item
			$P_0: (1+x)^1 \geq 1+x$\\
			$P_k:$ Assume $(1+x)^n \geq 1+nx$\\
			\begin{tabular}{l l}
				$P_{k+1}:$ & $ (1+x)^(n+1) \geq 1 + (n+1)x$\\
				& $(1+x)(1+x)^n \geq 1 + xn + x$\\
				& $(1+x)(1+x)^n \geq (1+x) + nx$\\
				& $(1+x)^n + x(1+x)^n \geq x + 1 + nx$\\
			\end{tabular}\\
			$x(1+x)^n \geq x$, because $(1+x)^n \geq 1$, so if $(1+x)^n \geq 1+nx$, then $(1+x)^n + x(1+x)^n \geq x + 1 + nx$.\\
		\item
			\begin{enumerate}
				\item
					$s(k)$ is a function.
					$dom(s) = \mathbb{N} \geq 1$\\
					$codom(s) = \mathbb{R} > 0$\\
				\item
					\begin{tabular}{l l}
						$e-s(k)$ & $ = \sum\limits_{i=0}^{\infty} \frac{1}{i!} - \sum\limits_{i=0}^{k} \frac{1}{i!}$\\
						& $= \sum\limits_{i=k+1}^{\infty} \frac{1}{i!} $\\
						& $= \frac{1}{(k+1)!} + \frac{1}{(k+2)(k+1)!} + \frac{1}{(k+3)(k+2)(k+1)!}$\\
					\end{tabular}\\
					where $n > 1$, $\frac{1}{k+n} < \frac{1}{k+1}$ therefore \\
					$= \frac{1}{(k+1)!} + \frac{1}{(k+1)(k+1)!} + \frac{1}{(k+1)^2(k+1)!} >$ the previous statement. \\
					therefore $e-s(n) < \frac{1}{(n+1)!}(\frac{1}{(n+1)} + \frac{1}{(n+1)^2} + ... )$\\	
					$\frac{1}{k!} < \frac{1}{(k+1)!} \frac{1}{(k+1)^{k+1}}$ because $k! > (k+1)!$ and $\frac{1}{(k+1)!} < 1$ and $\frac{1}{(k+1)^{k+1}}$.\\
				\item
					Because an infinite geometric series $\sum\limits_{i=0}^{n}a*r^n$ can be written as $\frac{a}{1-r}$, the infinite geometric series in this problem can be written as.\\
					\begin{tabular}{l l}
						$e - s(k)$ & $< \frac{1}{(k+1)!}\frac{1}{1-\frac{1}{k+1}}$\\
						& $< \frac{1}{(k+1)!}\frac{1}{\frac{k+1-1}{k+1}}$\\
						& $< \frac{1}{(k+1)!}\frac{1}{\frac{k}{k+1}}$\\
						& $< \frac{1}{(k+1)!}\frac{1}{\frac{k+1}{k}}$\\
						& $< \frac{1}{k*k!}$\\
					\end{tabular}\\
				\item
					If $r = \frac{m}{n}$,  $n! e = n! \frac{m}{n} = (n-1)! m$\\
					and $n! s(n) = n! \sum\limits_{i=0}^n \frac{1}{i!}$ $=$ $\sum\limits_{i=0}^n \frac{n!}{i!}$. At this point, the denominator always cancels out with a portion of the numberator, because $n \geq i$.
				\item $e-s(n) < \frac{1}{n!*n}$ where $n \epsilon \mathbb{N}$, so $n!(e-s(n)) < \frac{1}{n}$, which means $n!(e-s(n))$ ranges from $\{\frac{1}{1},\frac{1}{2}, ..., \frac{1}{\infty} = 0 \}$, or from [0,1] and $n!(e-s(n))$ must be positive because s(n) is smaller than e and n is a natural number, therefore it ranges from [0,1]\\
					however $n!(e-s(n)) = n!*e - n!(s(n))$ and both of those are integers if $e=\frac{m}{n}$, so $n!(e-s(n))$ must be an integer.
					therefore $n!(e-s(n))$ must be an integer between 0,1
			\end{enumerate}
	\end{enumerate}
\end{document}

