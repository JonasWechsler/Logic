\documentclass{article}
\usepackage[utf8]{inputenc}
\usepackage[margin=1in]{geometry}
\usepackage{enumerate}
\usepackage{amsmath}
\usepackage{amssymb}
\usepackage{amstext}
\usepackage{array}
\newcommand{\thf}{\rule{\textwidth}{.4pt}}

\title{Submission 4.1}
\author{Jonas Wechsler}
\date{November 2014}

\begin{document}
	\maketitle
	\begin{enumerate}
		\item
			$P_0:$ If only one student has a black card, they know that they are the only one with a black card because they do not see anybody else with a black card.\\
			Assume that it takes exactly k repetitions for k students with black cards.\\

		\item
			$P_0: 1^3 + 5*1 = 6, \frac{6}{6} = 1 \land 1 \epsilon \mathbb{N}$\\
			$P_k:$ Assume $k^3 + 5k$ is divisible by 6.\\
			\begin{tabular}{l l}
				$P_{k+1}:$ & $ (k+1)^3 + 5(k+1)$\\
				& $ k^3 + 3k^2 + 3k + 1 + 5k + 5$\\
				& $ k^3 + 3k^2 + 8k + 6$\\
				& $ (k^3 + 5k) + 3k(k + 1) + 6$\\
			\end{tabular}\\
			A sum of numbers that are multiples of 6 is also a multiple a multiple of 6. We have already assumed that $k^3 + 5k$ is a multiple of 6 and we have already shown that 6 is a multiple of 6. If k is even, then $3k$ is a multiple of 6, so $3k(k + 1)$ is also a multiple of 6. If k is odd, then $k+1$ is even and $3(k+1)$ is a multiple of 6. So, $3k(k+1)$ is a multiple of 6. Because the expression $(k^3 + 5k) + 3k(k+1) + 6$ is a multiple of 6, it is also divisible by 6.
		\item
			$P_0:$ 6 can be represented with 3 2 cent coins.\\
			Assume n cents can be represented with 2 and 7 cent coins.\\
			There are 2 cases for n+1. If n+1 is even, it can be represented with 2 cent coins. If n+1 is odd, then n+1-7 is even, and can be represented with 2 cent coins and a 7 cent coin.\\
		\item
			$P_0: |1| \leq |1|$\\
			$P_k:$ Assume $|x_1 + x_2 + ... + x_n| \leq |x_1| + |x_2| + ... + |x_n|$\\
			$P_{k+1}: |x_1 + x_2 + ... + x_n + x_{n+1}| \leq |x_1| + |x_2| + ... + |x_n| + |x_{n+1}|$\\
			In the case that $x_{n+1}$ is negative, $|x_1 + ... + x_n| \geq |x_1 + ... + x_{n+1}|$, and $|x_1| + ... + |x_n| \leq |x_1| + ... + |x_{n+1}|$, meaning that $|x_1 + ... + x_{n+1}| \leq |x+1| + ... + |x_{n+1}|$. In the case that $x_{n+1}$ is zero, $P_{k+1} = P_{k}$. In the case that $x_{n+1}$ is positive, $|x_1 + ... + x_n| + |x_{n+1}| = |x_1 + ... + x_{n+1}|$, so $|x_1 + ... + x_{n+1}| \leq |x_1| + ... + |x_{n+1}|$.
		\item
			$P_0: (1+x)^1 \geq 1+x$\\
			$P_k:$ Assume $(1+x)^n \geq 1+nx$\\
			\begin{tabular}{l l}
				$P_{k+1}:$ & $ (1+x)^(n+1) \geq 1 + (n+1)x$\\
				& $(1+x)(1+x)^n \geq 1 + xn + x$\\
				& $(1+x)(1+x)^n \geq (1+x) + nx$\\
				& $(1+x)^n + x(1+x)^n \geq x + 1 + nx$\\
			\end{tabular}\\
			$x(1+x)^n \geq x$, because $(1+x)^n \geq 1$, so if $(1+x)^n \geq 1+nx$, then $(1+x)^n + x(1+x)^n \geq x + 1 + nx$.\\
		\item
			\begin{enumerate}
				\item
					$s(k)$ is a function.
					$dom(s) = \mathbb{N} \geq 1$\\
					$codom(s) = \mathbb{R} > 0$\\
				\item
					\begin{tabular}{l l}
						$P_0:$ & $ e - s(1) < \frac{1}{(1+1)!}(1)$\\
						& $e - \frac{1}{1} - \frac{1}{1} < frac{1}{2}$\\
						& $e - 2 < \frac{1}{2}$\\
						\\
						$P_k$ & $ e - s(k) < \frac{1}{(k+1)!}(1+\frac{1}{k+1} + \frac{1}{(k+1)^2} + ...)$\\\\
						$P_{k+1}$ & $ e - s(k+1) < \frac{1}{(k+1)!} (1+\frac{1}{k+1}+\frac{1}{(k+1)^2} + ... + \frac{1}{(k+1)^{k+1}})$\\
						& $ e - s(k) - \frac{1}{k!} < \frac{1}{(k+1)!}(1+...) + \frac{1}{(k+1)!} \frac{1}{(k+1)^{k+1}}$\\
					\end{tabular}
					$\frac{1}{k!} < \frac{1}{(k+1)!} \frac{1}{(k+1)^{k+1}}$ because $k! > (k+1)!$ and $\frac{1}{(k+1)!} < 1$ and $\frac{1}{(k+1)^{k+1}}$.\\
				\item
			\end{enumerate}

	\end{enumerate}
\end{document}

