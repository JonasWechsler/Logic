\documentclass{article}
\usepackage[utf8]{inputenc}
\usepackage[margin=1in]{geometry}
\usepackage{enumerate}
\usepackage{amsmath}
\usepackage{amssymb}
\usepackage{amstext}
\usepackage{array}
\newcommand{\thf}{\rule{\textwidth}{.4pt}}

\title{Submission 3.1}
\author{Jonas Wechsler}
\date{Cheshvan 2014}

\begin{document}
	\maketitle
	\begin{enumerate}
		\item
			$Lxy:$ x loves y\\
			No
		\item
			$Mxy:$x is the mother of y\\
			No
		\item
			$Cxy:$ x is the child of y\\
			Yes
		\item
			$Hxy:$ x is the height of y\\
			No
		\item
			$Hxy:$ y is the height of x\\
			No
		\item
			$Fxy:$ y is the first (legal) wife of x\\
			No (not everybody is married)
		\item
			\begin{tabular}{l l}
				1. $\forall _x (f(x,0)=x)$ & $P_1$\\
				2. $\forall _x (f(x,g(x)) = 0)$ & $P_2$\\
				3. $\forall _x \forall _y f(x,y) = f(y,x)$ & $P_3$\\
				4. $\exists _x (f(x,0) = x)$ & EG(1)\\
				5. $\exists _x (f(x,g(x)) = 0$ & EG(2)\\
				6. $\exists _x \exists _y(f(x,y)=f(y,x))$ & EG(3)\\
				7. $f(0,g(0)) = 0$ & EI(5)\\
				8. $f(0,g(0)) = f(g(0),0)$ & EI(6)\\
				9. $f(g(0),0) = 0$ & Subst(7,8)\\
				10. $\exists _a (f(g(a),0) = a)$ & EG(9)\\
				11. $\exists _a (a = g(a))$ & Subst(4,10)\\
			\end{tabular}
			\begin{tabular}{l l}
				1. $\forall _x (f(x,0)=x)$ & $P_1$\\
				2. $f(0,0) = 0$ & $UI(1)$\\
				3. $\exists _y(f(y,y)=y)$ & $EG(2)$\\
			\end{tabular}
		\item
			$f:\mathbb{R}^2 \implies \mathbb{R}$\\
			$g:\mathbb{R} \implies \mathbb{R}$\\
			$f(x,y) = x+y$\\
			$g(x) = -x$\\
		\item
			Neither	
		\item
			Injective but not surjective
		\item
			Both
		\item
			Surjective but not injective
		\item
			Both\\
			Every natural number has a unique representation in roman numerals, therefore $f$ is surjective. No two natural numbers have the same representation in Roman Numberals. Therefore $f$ is injective.
		\item
			Surjective but not injective\\
			Consider that $f(1)=a$ and $f(27)=a$ and $1 \neq 27$, so $f$ is not injective.  Additionally, and that the domain is $\mathbb{R}$, while the codomain is restricted to letters $a$ through $z$, and every element in B is pointed to by at least one element in A. Therefore $f$ is surjective.
		\item
			Injective but not surjective\\
$\neg \exists x (f(x) = 1 \land x \in Domain(f))$. Therefore $f$ is not surjective.
If $f(x_1) = y = f(x_2)$, then $2 x_1 = 2 x_2$ so $x_1 = x_2$. Therfore $f$ is injective.
		\item
			Both\\
Given any $y \in Codomain(f)$, $f(y - 1) = y$, so $\forall y \exists x (y \in Codomain(f) \implies (f(x) = y \land x \in Domain(f)))$, so $f$ is surjective.
If $f(x_1) = y = f(x_2)$, then $x_1 - 1 = x_2 - 1$ so $x_1 = x_2$. Therefore, $f$ is injective.	
		\item
			Neither\\
$\neg \exists x (f(x) = -1 \land x \in Domain(f))$, so $f$ is not surjective.
$f(-2) = 4 = f(2)$, but $2 \neq -2$, so $f$ is not injective.
		\item
			Both\\
Given any $y \in Codomain(f)$, $f(\sqrt[3]{y}) = y$. Therefore $\forall y \exists x (y \in Codomain(f) \implies (f(x) = y \land x \in Domain (f)))$. Therefore $f$ is surjective.
If $f(x_1) = y = f(x_2)$, then $x_1^3 = x_2^3 $. $f(\sqrt[3]{x_1})=f(\sqrt[3]{x_2})$, where $\sqrt[3]{x}$ is a function, so $x_1 = x_2$. Therfore $f$ is injective.
		\item
			Injective but not surjective\\
$\neg \exists x (f(x) = -1 \land x \in Domain(f))$. Therefore $f$ is not surjective.
$f(x_1) = f(x_2) \implies f(ln(x_1)) = f(ln(x_2))$, where $ln(x)$ is a function,$ \implies e^{ln(x_1)} = e^{ln(x_2)} \implies x_1 = x_2$. Therefore $f$ is injective.
		\item
			Neither\\
			The number 2 exists in the Codomain of f, but $\sin^{-1} 2$ does not exist. Therefore, not all things in the Codomain of f exist in the image, so if is not surjective. $ f(0) = 0 = f(2 \pi) $. Therefore $f$ is not injective.

		\item
			Surjective but not injective\\
Given any $y \in Codomain(f)$, $f(\sin^{-1} y) = y$. Therefore $\forall y \exists x (y \in Codomain(f) \implies (f(x) = y \land x \in Domain(f)))$. Therefore $f$ is surjective.

$ f(0) = 0 = f(2 \pi) $. Therefore $f$ is not injective.

		\item			
			Surjective but not injective\\
Given any $y \in Codomain(f)$, $f(y) = y$. Therefore $\forall y \exists x (y \in Codomain(f) \implies (f(x) = y \land x \in Domain(f)))$. Therefore $f$ is surjective.

$f(.2) = 0 = f(.1)$, but $\frac{1}{2} \neq \frac{1}{3}$.

	\end{enumerate}
\end{document}

