\documentclass{article}
\usepackage[utf8]{inputenc}
\usepackage[margin=1in]{geometry}
\usepackage{enumerate}
\usepackage{amsmath}
\usepackage{amstext}
\usepackage{array}
\newcommand{\thf}{\rule{\textwidth}{.4pt}}

\title{Submission 1.2}
\author{Jonas Wechsler}
\date{August, September 2014}

\begin{document}

	\maketitle
	\section{Arguments in English}
	\begin{enumerate}
		\item[1]
			\begin{tabular}{>{$}c<{$} >{$}c<{$}}
				P \lor M & P_1\\ 
				\lnot P & P_2\\
				M & DS(1,2)
			\end{tabular}
		\item[4]
			\begin{tabular}{>{$}c<{$} >{$}c<{$}}
				R \implies F & P_1 \\
				F \implies \lnot R & P_2 \\
				\lnot F \implies \lnot R & ContraPos(1)\\
				F \lor \lnot F & Taut \\
				\lnot R \lor \lnot R & CD(2,3,4) \\
				\lnot R & Rep \\
			\end{tabular}
		\item[6]
			\begin{tabular}{>{$}c<{$} >{$}c<{$}}
				C \implies P & P_1 \\
				\lnot P & P_2 \\
				\lnot P \implies \lnot C & ContraPos(1) \\
				\lnot C & MP\\
			\end{tabular}
		\item[7]
			\begin{tabular}{>{$}c<{$} >{$}c<{$}}
				A \implies \lnot G & P_1 \\
				M \implies A & P_1 \\
				\lnot \lnot G \implies \lnot A & ContraPos(1) \\
				G \implies \lnot A & DN \\
				\lnot A \implies \lnot M & ContraPos(2) \\
				G \implies \lnot M & HS(4, 5) \\
			\end{tabular}
		\item[10]
			\begin{tabular}{>{$}l<{$} >{$}l<{$}}
				M \iff H & P_1 \\
				M \land \lnot H & P_2 \\
				M & Simp(2) \\
				H & DefOfEq(3) \\
				\lnot H & Simp(2) \\
				H \lor E & Add(4) \\
				E & DS(5,6)\\
			\end{tabular}\\
		\item[12]
			\begin{tabular}{>{$}l<{$} >{$}l<{$}}
				O \iff E & P_1 \\
				\lnot H \implies \lnot E & P_2 \\
				H \implies \lnot E & P_3 \\
				H \lor \lnot H & Taut \\
				\lnot E \lor \lnot E & CD(2,3,4) \\
				\lnot E & Rep \\
				\lnot E \lor \lnot O & Add \\
			\end{tabular}
		\item[13]
			\begin{tabular}{>{$}l<{$} >{$}l<{$}}
				M \implies \lnot N & P_1 \\
				\lnot M \implies R & P_2 \\
				N \and \lnot R & P_3 \\
				\lnot \lnot N \implies \lnot M & ContraPos(1) \\

				\lnot M \land \lnot \lnot M 
			\end{tabular}
		\item[14]
			\begin{tabular}{>{$}l<{$} >{$}l<{$}}
				P \implies (F \implies E) & P_1 \\
				\lnot E & P_2 \\
				\lnot P \lor (F \implies E) & CDis(1) \\
				\lnot P \lor (\lnot E \implies \lnot F) & ContraPos(3) \\
				\lnot P \lor \lnot F & MP(2,4)\\
			\end{tabular}
		\item[15]
			\begin{tabular}{>{$}l<{$} >{$}l<{$}}
				L \implies Y & P_1 \\
				\lnot L \implies (S \lor R) & P_2 \\
				\lnot (S \lor R) \implies L & ContraPos(2) \\
				\lnot (S \lor R) \implies Y & HS(1,3) \\
				\lnot \lnot (S \lor R) \lor Y & CDis(3) \\ 
				(S \lor R) \lor Y & DN(4) \\
				R \lor (Y \lor S) & Commute \lor (5) \\
				\lnot R \implies (Y \lor S) & CDis(6) \\
				\lnot R \implies (\lnot Y \implies S) & CDis(7)\\
			\end{tabular}
	\end{enumerate}
	\section{Claims}
	\begin{enumerate}
		\item[13]
			The statement $(A \land B) \land C$ is true if and only if A, B, and C are true. The statement $A \land (B \land C)$ is true if and only if B, C, and A are all true. Therefore, the statements $(A \land B) \land C$ and $A \land (B \land C)$ are equivalent.\\
			The statement $(A \lor B) \lor C$ is false if and only if A, B, and C are false. The statement $A \lor (B \lor C)$ is false if and only if B, C, and A are all false. Therefore, the statements are equivalent. \\
			$(F \implies T) \implies F$ is False, while $F \implies (T \implies F)$ is True. Therefore, $A \implies (B \implies C)$ and $(A \implies B) \implies C$ are equivalent.\\
			The statement $(A \iff B) \iff C$ is true if and only if A, B, and C are all equivalent. (All either true or false.) The statement $A \iff (B \iff C)$ is true if and only if A, B, and C are all equivalent. Therefore, the two statements are equivalent. \\
		\item[18]
			The argument \begin{tabular}{c}A\\B\\\hline C\end{tabular} will always be valid, because there is no case in which A and B will be true and C will be false, since there is no case in which C is false. \\
		\item[19]
			$(A \lor B)\iff A \iff B$, since $A \iff B$ and $(A \lor A) \iff A$. \\
		\item[20]
			
		\item[21]

		\item[22]
	\end{enumerate}
\end{document}

