\documentclass{article}
\usepackage[utf8]{inputenc}
\usepackage{enumerate}
\usepackage{amsmath}
\usepackage{array}

\title{Get logic'd}
\author{Jonas Wechsler}
\date{August 2014}

\begin{document}

	\maketitle

	\section{Day 1}
	If I am deceived, I exist.
	\\ \underline{If I am not deceived, I do not exist.}
	\\ I exist.
	\section{Day 2: Validity}
	An argument is \textbf{valid} if and only if the truth of its premises guarantees the truth of the conclusion. (i.e. there is no case in which all premises are true and the conclusion is false)
	\\An argument is \textbf{vacuously valid} if and only if it is valid and has inconsistent premises.
	\\An argument is \textbf{sound} if it is both valid and the premises are true.
	\\An argument is \textbf{consistent} if and only if there is a case that all sentences are true.
	\\An argument is \textbf{defeasibly valid} if and only if, if the premises are true, it's reasonable to expect the conclusion to be true as well.
	\\A \textbf{counterexample} to an argument is a circumstance in which the premises are true and the conclusion is false.
	\\A proof technique is called \textbf{formal} if and only if it can be carried out by a computer.
	\\\textbf{Theorem-argument exchange theorem}?
	\section{Day 3: Operators}
	If I am deceived, then I exist.\\
	If I am not deceived, then I exist\\
	A: I am deceived.\\
	B: I exist.\\
	\\
	\underline{Logical Operators}\\
	negation $\lnot$ not\\
	biconditional $\iff$ if and only if, necessary and sufficient, exactly when\\
	conditional $\implies$ if-then, A only if B\\
	disjunction $\lor$ or\\
	conjunction $\land$ and\\
	\\
	$((p \implies q) \land  p)\implies q$ Modus ponens\\
	$((p \implies q) \land  !q)\implies !p$ Modus tollens\\
	\section{Day 4: More definitions and truth tables}
	\textbf{Case}: An assignment of truth values to atomic sentences.\\
	\\
	\\If Einstein's theory of relativity is correct, light bends in the vicinity of the sun.
	\\\underline{Light bends in the vicinity of the sun.}
	\\Einstein's theory of relativity is correct.
	\\
	$E$:Einstein's theory of relativity is correct.\\
	$L$:Light bends in the vicinity of the sun.\\
	\\
	\begin{tabular}{l | l || l | l || l}
		E & L & $E \implies L$ & L & E\\ \hline
		T & T & T & T & T\\
		T & F & F & F & T\\
		F & T & T & T & F\\
		F & F & T & F & F\\
	\end{tabular}\\
	\\
	\begin{tabular}{l | l | l || l | l || l}
		A & G & L & $A \implies \lnot G$ & $L \implies A$ & $G \implies \lnot L$\\ \hline
		T & T & T & F & T & F\\
		T & T & F & F & F & T\\
		T & F & T & T & T & T\\
		T & F & F & T & F & T\\
		F & T & T & T & F & F\\
		F & T & F & T & T & T\\
		F & F & T & T & F & T\\
		F & F & F & T & T & T\\
	\end{tabular}\\
	\\
	\begin{tabular}{l | l || l | l | l | l | l}
		A & B & $\lnot A$ & $A \land B$ & $A \lor B$ & $A \implies B$ & $A \iff B$\\ \hline
		T & T & F & T & T & T & T\\
		T & F & F & F & T & F & F\\
		F & T & T & F & T & T & F\\
		F & F & T & F & F & T & T\\
	\end{tabular}\\
	\\
	\section{Day 5: To prove validity}
	Raymond Smullyan\\
	\subsection{Formal Ways}
	Truth Tables, see Day 4\\\\
	\subsection{Informal Ways}
	\textbf{Conditional Proof}\\
	Assume all premises are true, apply definitions, past theorems. Pure light of reason!.\\
	Show the conclusion is also true.\\
	\textbf{Proof by contradition}\\
	Assume argument is invalid (i.e., there's a counter example, i.e. there's a case where all prepositions are true and c is false.)\\
	\section{Day 6:Logical equivalencies}
	Don't (drink and drive).\\
	Don't drink or don't drive.\\
	Drink nand drive.\\
	Drink only if not driving.\\
	\\
	\subsection{Logical Equivalencies}
	\textbf{DeMorgan's Law:}$\lnot(A \land B)\iff(\lnot A \lor \lnot B)$\\
	\textbf{Conditional Disjunction:}$(A \implies B)\iff(\lnot A \lor B)$\\
	\textbf{Contra-positive:}$(A \implies B)\iff(\lnot B \implies \lnot A)$\\
	\textbf{Definition of Equivalence:}$(A \iff B)\iff((A \implies B)\land(B \implies A))$\\
	\textbf{Distribution of And over Or:}$(A \land (B \lor C)) \iff ((A \land B) \lor (A \land C))$\\
	\textbf{Distribution of Or over And:}$A \lor (B \land C) \iff ((A \lor B)\land(A \lor C))$\\
	\subsection{Some definitions and stuff}
	\textbf{Main operator: }The most "outer" operator in a logical statement.\\
	\textbf{Tautology: }Always true; Logically equivalent.\\
	Logically equivalent : Tautology :: Contradictory sentences : Contradiction\\
	\textbf{Proofs of iff: }Prove $A \implies B$ and $B \implies A$ to prove $A \iff B$.
	\section{Day 7:Rules of Inference}
	\begin{tabular}{>{$}l<{$}}
		A \\ \hline
		(A \implies B)\implies B\\
		(A \lor B)
	\end{tabular}\\\\
	\subsection{Natural Deduction}
	\begin{tabular}{>{$}l<{$} >{$}l<{$} >{$}l<{$}}
		1. & A \implies \lnot G & P_1 \\
		2. & L \implies A & P_2 \\
		3. & G \implies \lnot A & ContraPos(1)\\
		4. & \lnot A \implies \lnot L & ContraPos(2)\\
		6. & G \implies \lnot L & Hypothetical Syllogism
	\end{tabular}\\
	\begin{tabular}{>{$}l<{$} >{$}l<{$}}
		M \iff H & P_1 \\
		M \land \lnot H & P_2 \\
		M & Simp(2) \\
		H & DefOfEq(3) \\
		\lnot H & Simp(2) \\
		H \lor E & Add(4) \\
		E \\ 	
	\end{tabular}\\
	\begin{tabular}{>{$}l<{$} >{$}l<{$}}
		O \iff E & P_1 \\
		\lnot H \implies \lnot E & P_2 \\
		H \implies \lnot E & P_3 \\
		H \lor \lnot H & Taut \\
		\lnot E \lor \lnot E & CD(2,3,4) \\
		\lnot E & Rep \\
		\lnot E \lor \lnot O & Add \\
	\end{tabular}
\end{document}

